% texsalq.tex -- the template for typesetting theses at ESALQ - USP - Piracicaba
%---------------------------------------------------------------------
% Luciano R Silveira  Jul. 2014
%
% Note: This template was based on ctustyle.tex -- the template for typesetting theses at CTU in Prague, by Petr Olsak

\def\esalqstyleversion{beta(b) Out. 2014}

\message{TeXSALQ: Dissertacoes e teses na ESALQ - USP, 
         v. <\esalqstyleversion>}

%%% Testing versions of opmac

\def\esalqstyleERR#1{\message{ERROR -- #1.}\expandafter\end}

\ifx\pdfoutput\undefined
   \esalqstyleERR {pdfTeX isn't detected, use ``pdftex \jobname'' command}%
\fi
\pdfoutput=1

%\def\tmp#1#2\end{\if$#2$\else 
%   \esalqstyleERR {csplain doesn't read UTF-8 encoding, may be it is an old version}\fi}%
%\tmp č\end

\newread\testinput
\def\testfile#1#2{\openin\testinput=#1
   \ifeof\testinput \esalqstyleERR {#1 not found, install it from #2}\fi
   \closein\testinput
}
\testfile{macros/opmac.tex}{petr.olsak.net/opmac.html}
%\testfile{macros/ams-math.tex}{petr.olsak.net/opmac.html}
%\testfile{lmfonts.tex}{petr.olsak.net/csplain.html}

\def\totlist{} \def\toflist{}  % List of tables and figures

\def\Xseccc#1#2#3{\addto\toclist{\tocline{0}{\rm}{#1}{#2}{#3}}} % Third level sections
\def\Xtab#1#2#3{\addto\totlist{\totline{#1}{#2}{#3}{t}}}
\def\Xfig#1#2#3{\addto\toflist{\tofline{#1}{#2}{#3}{f}}}

% Redefinicao da funcao Xbib do OPmac, com um novo nome.
%Deve ser declarada antes do input, senao nao funciona corretamente
\def\Xbibcite#1#2#3{\sdef{bib:#1}{\bibnn{#2}&}
                    \sdef{bibC:#1}{\bibnn{#2}&}
                    \sdef{bibD:#1}{\bibnn{#2}&}
                    \sdef{bibE:#1}{\bibnn{#2}&}
                    \sdef{bibF:#1}{\bibnn{#2}&}
                    \if^#3^\else\sdef{bim:#2}{#3}\fi\def\lastbibnum{#2}}
\let\normalmath\relax % pro opmac nao carregar o ams-math

\let\regfont\relax % pro opmac nao carregar a fonte csbxti10
% O bloco abaixo deve ser mantido. Definicoes necessarias devido ao nao carregamento da fonte tcheca csbxti10 do opmac.
\def\letfont#1#2{\ifx#2=\expandafter\letfont\expandafter#1\else
  \expandafter\font\expandafter#1\expandafter\rfontskipat\fontname#2 \relax\space \fi}
\def\rfontskipat#1{\ifx#1"\expandafter\rfskipatX\else\expandafter\rfskipatN\expandafter#1\fi}
\def\rfskipatX #1" #2\relax{"\whichtfm{#1}"}  \def\rfskipatN #1 #2\relax{\whichtfm{#1}}
\def\sizespec{}   \def\whichtfm#1{#1}
\def\resizefont#1{\letfont#1#1\sizespec}

\input macros/opmac


\ifx\remskip\undefined
   \esalqstyleERR {OPmac older than Jun. 2013. Upgrade from petr.olsak.net/opmac.html}%
\fi

%%% Useful Macros

\def \newpage {%
  \par%
  \vfil%
  \break}

\def\cleardoublepage{\newpage\ifodd\pageno\else% se for impar nao faz nada
                     \null\newpage\fi}% ou \hbox{}\newpage, 'a moda LaTeX

% Implementacao de \settodim do LaTeX (convertendo p plain TeX) 
% \settowidth armazena em \mylen o comprimento de um texto
\newbox\tempboxa \newdimen\mylen
\def\settodim#1#2#3{%
   \setbox\tempboxa\hbox{{#3}}% Nao sei muda hbox pra vbox para \dp e \ht
   \global#2=#1\tempboxa%
%   \leavevmode% in case it's the first thing in a paragraph
 %  \box\tempboxa% serve para manter o texto no pdf. Aqui no caso nao eh necessario
}
\def\settoheight#1#2{\settodim{\ht}{#1}{#2}}
\def\settodepth#1#2{\settodim{\dp}{#1}{#2}}
\def\settowidth#1#2{\settodim{\wd}{#1}{#2}}

%Uma implementacao de \contentspush do pkg titlesec do LaTeX (nao esta igual)
\def\contentspush#1{%
%  \bgroup
  \settowidth{\mylen}{#1}
  \advance\leftskip\the\mylen%
  \par\noindent
  \leavevmode\llap{#1}\ignorespaces
%  #2\par\egroup
}
% Uma outra versao mais enxuta, faz a mesma coisa que o \contentspush, mas usa o parindent
\def\ctpush#1{%
%  \bgroup
  \settowidth{\mylen}{#1\enspace} % width with \enspace, because the \item macro add a \enspace
  \parindent=\mylen
  \item{#1}
  %#2\par\egroup
} %obs: contentspush nao foi usado diretamente aqui

\def\nohyphens{\pretolerance=9999 \tolerance=9999
               \hyphenpenalty=9999 \exhyphenpenalty=9999\relax} % para nao hifenizar
% Ha outros meios de controlar hifenizacao:
% http://tex.stackexchange.com/questions/31301/how-to-reduce-the-number-of-hyphenation

%\newskip\flushglue \flushglue = 0pt plus 1fil %0pt plus4em
\newskip\flushgluel \flushgluel=0pt plus 1fil%
\newskip\flushgluer \flushgluer=0pt plus 1fil% Para ter controle independente

\def\raggedcenter{\leftskip=\flushgluel \rightskip=\flushgluer%
  \parfillskip=0pt \spaceskip=.3333em \xspaceskip=.5em
  \parindent=0pt \nohyphens}
%http://tex.stackexchange.com/questions/45783/how-can-i-break-the-rows-of-an-halign-nicely-across-pages
%http://tug.org/pipermail/texhax/2009-February/011836.html
% latex.ltx file

\def\raggedleft{\leftskip=\flushgluel \rightskip=0pt%
  \parfillskip=0pt \spaceskip=.3333em \xspaceskip=.5em%
  \parindent=1.5cm \nohyphens} % adaptei
% outras ideias de ragged: http://mirror.unl.edu/ctan/macros/plain/contrib/tugboat/tugboat.cmn


% Livro do Knut p. 311
%\def\frac#1/#2{\leavevmode\kern.1em
\def\frac#1/#2 {\leavevmode\kern.1em
\raise.5ex\hbox{\the\scriptfont0 #1}\kern-.1em
%/\kern-.15em\lower.25ex\hbox{\the\scriptfont0 #2}}
/\mkern-1mu\hbox{\the\scriptfont0 #2}} % same that nicefrac LaTeX Package

% Sample usage: $\frac1/4 $


%%% Declaration commands:

\newtoks\title
\newtoks\author
\newtoks\supervisor   \let\supervisors=\supervisor
\newtoks\city
\newtoks\date
\newtoks\titarea
\newtoks\concarea
\newtoks\authorinfo
\newtoks\workname
\newtoks\pagetwo
\newtoks\titleEN     \newtoks\titlePT
\newtoks\abstractEN  \newtoks\abstractPT
\newtoks\keywordsEN  \newtoks\keywordsPT
\newtoks\dedication
\newtoks\thanks
\newtoks\epigraph

%%% Mandatory declaration commands

\def\mandatorydecl#1{\if&\the#1&%
   \esalqstyleERR {the mandatory item \string#1 is undeclared or empty}%
   \fi
}
\def\makefront{%
   \ifx\mainlanguage\undefined
      \esalqstyleERR {The \string\worktype[<type>/<lang>] command 
                    is missing before \string\makefront}\fi
   \everypar={}
   \mandatorydecl\title
   \mandatorydecl\author
   \mandatorydecl\date
   \mandatorydecl\supervisor 
   \ifnum\worktypenum>0
      \mandatorydecl\abstractEN
%      \def\tmp{EN}\ifx\mainlanguage\tmp
%         \if&\the\abstractSK&\mandatorydecl\abstractCZ \fi
%      \fi
      \def\tmp{PT}\ifx\mainlanguage\tmp \mandatorydecl\abstractPT \fi
      \mandatorydecl\titarea
      \mandatorydecl\concarea
   \else \mandatorydecl\workname
   \fi
}

\everypar={\normaltypingdenied}

\def\normaltypingdenied{%
   \esalqstyleERR{Text outside parameters on line \the\inputlineno.
                Use \noexpand\makefront first}\relax}


%%% Automatically generated multilingual texts:

\sdef{mt:chap:pt}{Cap{\'\i}tulo}
\sdef{mt:t:pt}{Tabela}
\sdef{mt:f:pt}{Figura}

\sdef{lan:7}{pt}  \sdef{lan:17}{pt}  \sdef{lan:117}{pt}

\def\slet#1#2{\expandafter\let\csname#1\expandafter\endcsname\csname#2\endcsname}
\def\mtdef#1#2#3#4{\sdef{mt:#1:en}{#2} \sdef{mt:#1:\pt}{#3}
  \if$#4$\slet{mt:#1:sk}{mt:#1:\pt}
  \else  \sdef{mt:#1:sk}{#4}
  \fi
}
\edef\pt{\csname lan:7\endcsname}

\mtdef {usp}   {University of S�o Paulo} 
               {Universidade de S�o Paulo} {} 

\mtdef {esalq}    {``Luiz de Queiroz'' College of Agriculture}
                  {Escola Superior de Agricultura ``Luiz de Queiroz''} {}

\mtdef {abstract}     {ABSTRACT}          {RESUMO}           {}
\mtdef {thanks}       {ACKNOWLEDGEMENT}   {AGRADECIMENTOS}   {} 
%\mtdef {keywords}     {Keywords}          {Palavras-chave}   {}
\mtdef {contents}     {CONTENTS}          {SUM�RIO}          {}
\mtdef {tables}       {LIST OF TABLES}    {LISTA DE TABELAS} {}
\mtdef {figures}      {LIST OF FIGURES}   {LISTA DE FIGURAS} {}
\mtdef {supervisor}   {Supervisor}        {Orientador}       {}
\mtdef {bibliography} {REFERENCES}        {REFER�NCIAS}      {}
\mtdef {appendix}     {APPENDIX}          {AP�NDICE}         {}  
\mtdef {appendixpl}   {APPENDIXES}        {AP�NDICES}        {}
%\mtdef {gradem}       {Master}            {Mestre}           {}
%\mtdef {graded}       {Ph.D.}             {Doutor}           {}

\mtdef {M} {Thesis submitted in partial fulfillment of requirements for the degree of Master in \the\titarea. Area of concentration: \the\concarea}   
           {Disserta��o apresentada para obten��o do t�tulo de Mestre em \the\titarea. �rea de concentra��o: \the\concarea}   {}
\mtdef {D} {Thesis submitted in partial fulfillment of requirements for the degree of Doctor in \the\titarea. Area of concentration: \the\concarea}     
           {Tese apresentada para obten��o do t�tulo de Doutor em \the\titarea. �rea de concentra��o: \the\concarea}   {}

%\def\keepacc#1{\slet{mt:#10:sk}{mt:#1:sk}\slet{mt:#10:\pt}{mt:#1:\pt}}
%\def\keepaccents{\keepacc{thanks}%
%  \keepacc{declaration}\keepacc{figures}\keepacc{specifi}}


%%% Worktype

\def\worktype[#1/#2]{%
   \isdefined{wt:#1}\iftrue \csname wt:#1\endcsname \relax
      \else \esalqstyleERR {Unknown \noexpand\worktype parameter}\fi
   \isdefined{wl:#2}\iftrue \csname wl:#2\endcsname \relax
      \else \esalqstyleERR {Unknown \noexpand\worktype parameter}\fi
}
\sdef{wt:O}{\chardef\worktypenum=0}
\sdef{wt:M}{\chardef\worktypenum=1}
\sdef{wt:D}{\chardef\worktypenum=2}

\sdef{wl:EN}{\def\mainlanguage{EN}\ehyph}
\sdef{wl:PT}{\def\mainlanguage{PT}\pthyph}

%%% Fonts

\let\LMTone\relax % If in english, change to CMTone
\input macros/mfontch

%--- Espacamento entre linhas nas fontes -----------------
%--- Medidas analisadas do MS Word
%
%-----Fonte 10
\def\tensingle{\baselineskip 11.50pt}
\def\tenonehalf{\baselineskip 17.32pt}
\def\tendouble{\baselineskip 23.14pt}

% --- Fonte 12
\def\twelvesingle{\baselineskip 13.84pt}
\def\twelveonehalf{\baselineskip 20.84pt}
\def\twelvedouble{\baselineskip 27.72pt}

% Italico negrito tamanho 12 (nao e' definido na macro fontch.tex)
\font\twelvebfit=cmbxti10 scaled 1200
\let\bfit=\twelvebfit

% -----------------------------------------------
%% Definicao da fonte italica para indices no modo matematico,
%%  relativo a fontes no tamanho 12 ---
%\font\eightit=ec-lmri8  % Ja definido na macro fontch.tex
\font\sixit=ec-lmri7 at 6pt
\scriptfont\itfam=\eightit
\scriptscriptfont\itfam=\sixit

%%% Colors

\def\Orange{\setcmykcolor{0 0.5 1 0}}
\def\Blue{\setcmykcolor{1 .43 0 0}}
\def\liBlue{\setcmykcolor{.2 .08 0 0}}
\def\liGrey{\setcmykcolor{0 0 0 0.13}}
\let\nBlue=\Blue 

\def\blackwhite{\let\Blue=\Grey \let\nBlue=\Black \let\Red=\Grey \let\liBlue=\liGrey}
\let\trysavetoner=\relax

\def\savetoner{\def\trysavetoner{%
  \ifx\drafttext\empty
     \message{WARNING: final (not \string\draft) version,
              \noexpand\savetoner ignored}
  \else
     \let\liBlue=\White
     \let\liGrey=\White % added 2014-11-02
  \fi 
}}

\hyperlinks{\Black}{\Black}
\def\tocborder{1 .8 0} 
\let\pgborder\tocborder
\let\citeborder\tocborder
\let\refborder\tocborder
\let\urlborder\tocborder

\ifx\localcolor\undefined  \let\locc=\relax \else \let\locc=\localcolor \fi


%%% Typesetting area

%\margins/pg fmt (left,right,top,bot)unit
\margins/2 a4 (30,20,30,20)mm % set margins
                              % for A4 paper

% Comportamento do texto na pagina, etc. Quase equivalente ao Sloppy no LaTeX.
\tolerance 1414  
\hbadness 1414   
\emergencystretch 2em %do opmac-d
\hfuzz 0.3pt    
\vfuzz \hfuzz   
%site: http://www.latex-community.org/forum/viewtopic.php?f=19&t=21170

\parindent 1.5cm \iindent=\parindent %iindent controla indentacao de itens, sumario, etc.

\let\firstnoindent=\relax %indenta sempre o primeiro paragrafo

\raggedbottom

%\nopagenumbers  % para nao colocar o numero da pagina no rodape, quando footline nao esta definido

\def\makeheadline{\vbox to0pt{\vskip-34pt
  \line{\vbox to10pt{}\the\headline}\vss}\nointerlineskip}
\def\makefootline{\baselineskip=34pt \lineskiplimit=0pt \line{\the\footline}}

\def\picdir{figuras/} % the directory with picture files

%%% Draft

{\tt\thefontsize[10] \global\let\ttt=\thefont}

\def\drafttext{}
\def\draft{\let\destbox=\draftdestbox 
     \def\drafttext{\vbox to0pt{\vss \llap{   
     \Grey \bf\thefontsize[11] Rascunho: \the\day. \the\month.
     \the\year\Black}}}}

\def\linespacing=#1#2 {\def\thelinespacing{#1#2}}  
\linespacing=1

\def\setlinespacing{%
   \ifdim\thelinespacing pt=1pt \else
      \ifx\drafttext\empty 
            \message{WARNING: final (not \string\draft) version, 
                     \noexpand\linespacing ignored}
         \else
            \baselineskip=\thelinespacing\baselineskip
      \fi 
   \fi
}
\def\draftdestbox[#1#2:#3]{\vbox to0pt{\kern-\destheight
   \pdfdest name{#1#2:#3} xyz\relax
   \if#1r\llap{\locc\Red\ttt[\ignoretocolon\csname lab:#3\endcsname]}\vss
   \else \if#1c\vss\llap{\locc\Red\ttt[\ignoretocolon\csname bib:\tmpb\endcsname] }\kern-\prevdepth
   \else \vss \fi\fi}}
\def\ignoretocolonA#1:{}
\def\ignoretocolon{\expandafter\expandafter\expandafter\ignoretocolonA\expandafter\string}

%%% PDF info data

\def\pdfinfodata{%
   {\def\TeX{TeX}\def\nl{ }%
    \ifx\pdfunidef\undefined
       \edef\tmp{/Author(\the\author)
            /CreationDate(\the\date)
            /Creator(TeX + ESALQstyle)
            /Title(\the\title)
            /Subject(\ifcase\worktypenum
                \the\workname \or \mtext{M}\or \mtext{D}\fi)
            /Keywords(\the\keywordsEN)}
%       \edef\toasciidata{\toasciidata}\expandafter \setlccodes \toasciidata{}{}%
%       \lowercase\expandafter{\expandafter\def\expandafter\tmp\expandafter{\tmp}}%
    \else
       \pdfunidef\tmpb{\the\author}\edef\tmp{/Author(\tmpb) /CreationDate(\the\date) }%
       \pdfunidef\tmpb{\the\title}%
       \edef\tmp{\tmp /Creator(TeX + ESALQstyle) /Title(\tmpb) }%
       \pdfunidef\tmpb{\ifcase\worktypenum
                \the\workname \or \mtext{M}\or \mtext{D}\fi}%
       \edef\tmp{\tmp /Subject(\tmpb) /Keywords(\the\keywordsEN)}%
    \fi
    \pdfinfo{\tmp}}
    \ifx\pdfunidef\undefined\else \keepaccents
       \let\insertoutlineI=\insertoutline
       \def\insertoutline##1{\pdfunidef\tmp{##1}\insertoutlineI{\tmp}}
    \fi
}
\addto\makefront{\pdfinfodata}

%%% Title page (cover) (Capa)

\def\titlepageA{\fourteenpoint\bf
               \baselineskip=17pt
               %\Blue
               \centerline{\mtext{usp}}\par
               \centerline{\mtext{esalq}}\par
               \vskip 5cm
               {\raggedcenter{\the\title}\par}
               \vskip 5cm
               \centerline{\the\author}\par
               \vskip 3cm
               \bgroup
                 \leftskip 7.8cm% 10.8 - 3
                 \tenpoint\noindent
                 \ifnum\worktypenum=0 \the\workname%
                 \else
                    \ifcase\worktypenum
                    \relax \or \mtext{M}\or \mtext{D}\fi
                 \fi
                 \par
               \egroup
               \vfill
               \centerline{\the\city}
               \centerline{\the\date}
}
\addto\makefront{%
   \bgroup \hbadness=4000 
   \pageno=-1 %\def\advancepageno{\global\advance\pageno by-1 }
   \footline={\hss\drafttext} \titlepageA \vfil\break
}

%%% Title page (folha de rosto)

\def\titlepageB{\twelvepoint
               \centerline{\the\author}\par
               \centerline{\the\authorinfo}\par
               \vskip 5cm
               {\raggedcenter{\bf \the\title}\par}
               \vskip 3.5cm
               \bgroup
                 \leftskip 8.26cm % 10.8 - 2.54
                 \tenpoint\noindent
                 \mtext{supervisor}:\par\noindent
		 %\mtext{supervisorD} {\bf \uppercase\expandafter{\the\supervisor}}
                 {\the\supervisor}
                 \vskip 3.5cm \noindent
                 \ifnum\worktypenum=0 \the\workname%
                 \else
                    \ifcase\worktypenum
                    \relax \or \mtext{M}\or \mtext{D}\fi
                 \fi
                 \par
               \egroup
               \vfill
               \centerline{\bf \the\city}
               \centerline{\bf \the\date}
}
\addto\makefront{%
   \pageno=-\pageno \advance\pageno by-1
   \pdfcatalog{/PageLabels<</Nums[0<</S/r>>\the\pageno<</S/D>>]>>}
   \egroup  % Ate aqui nao conta as paginas em arabico
   \pageno=1 \def\advancepageno{\global\advance\pageno by1 }
   \footline={\hss\drafttext}
   \titlepageB \vfil\break
}

%%% Page Two

\def\printpagetwo{\null\vskip0pt plus1fil {\parindent=0pt \the\pagetwo\endgraf}}
\addto\makefront{\printpagetwo \endgraf\break} 


%%% Headline definition

\addto\makefront{%
\headline={\ifodd\pageno \hfil\twelverm\folio % se for impar joga pra direita
           \else \twelverm\folio\hfil  % se for par joga pra esquerda
           \fi}
}

%%% Dedication page (Dedicat�ria) (Opcional)

\def\dedicpage{\bgroup\twelvepoint\it
               \twelveonehalf
               \line{}\vskip 1cm
               {\flushgluel=150pt plus4em \raggedleft{\the\dedication}\par}
               \egroup
}
\addto\makefront{%
   \if&\the\dedication&\else%
     \dedicpage \vfil\break\fi
}


%%% Thanks page (Optional)

\def\thankspage{\twelvepoint\twelveonehalf
                \centerline{\bf \mtext{thanks}}
                \vskip 1.5cm
                \the\thanks\par
}
\addto\makefront{%
   \if&\the\thanks&\else%
     \cleardoublepage\thankspage \vfil\break\fi
}

%%% Epigraph page (Optional)

\def\epigraphpage{\twelvepoint\twelveonehalf
                  \vskip 1cm
                  {\flushgluel=150pt plus4em \raggedleft{\the\epigraph}\par}
}
\addto\makefront{%
\if&\the\epigraph&\else%
   \cleardoublepage\epigraphpage \vfil\break\fi
}

%%% Contents

\def\tocline#1#2#3#4#5{{\rightskip=2\iindent
   \toclinehook \noindent %\llap{#2\toclink{#3}\enspace}%
      {\if&#3&\else\rm\toclink{#3}\enspace\fi}%
         {\rm#4\unskip}\nobreak\tocdotfill\pglink{#5}\nobreak\hskip-2\iindent\null\par}}%

\def\tocpages{\twelvepoint\twelveonehalf
   \centerline{\bf \mtext{contents}}
   \bigskip
   \maketoc
}
\addto\makefront{%
   \cleardoublepage\tocpages \vfil\break
}

%%% Abstract

\def\abstractpagePT{%
   {\twelvepoint\twelvesingle
   \pthyph
   \nonum\chap RESUMO\par
   \bigskip
   {\raggedcenter{\bf\the\titlePT}\par}
   \bigskip\noindent
   \the\abstractPT\par
   \bigskip\noindent
   Palavras-chave: \the\keywordsPT\par}
}
\def\abstractpageEN{%
   {\twelvepoint\twelvesingle
   \ehyph
   \nonum\chap ABSTRACT\par
   \bigskip
   {\raggedcenter{\bf\the\titleEN}\par}
   \bigskip\noindent
   \the\abstractEN\par
   \bigskip\noindent
   Keywords: \the\keywordsEN\par}
}

\addto\makefront{%
   \def\tmp{PT}\ifx\mainlanguage\tmp
     \let\titlePT=\title%
     \cleardoublepage\abstractpagePT
     \cleardoublepage\abstractpageEN \vfil\break
   \else
     \let\titleEN=\title%
     \cleardoublepage\abstractpageEN
     \cleardoublepage\abstractpagePT \vfil\break
   \fi
}

%%% Figures / Tables

\def\tofpages{
   \twelveonehalf%
   \nonum\chap \mtext{figures}\par
   \bigskip
   \maketof
   \par
}
\addto\makefront{%
     \cleardoublepage\tofpages \vfil\break %comentar essa linha se nao houver figuras
}
\def\totpages{
   \twelveonehalf%
   \nonum\chap \mtext{tables}\par
   \bigskip
   \maketot
   \par
}
\addto\makefront{%
     \cleardoublepage\totpages \vfil\break %comentar essa linha se nao houver tabelas
}

%%% Typessetting of the document:

\addto\makefront{%
   \twelvepoint\twelveonehalf
   \setlinespacing
   \trysavetoner
   \pdfcatalog{/PageMode /UseOutlines}
}

%%% Chapter, section

\def\titfont{\fourteenpoint\bf\baselineskip 27pt minus 1pt}
\def\chapfont{\twelvepoint\bf\baselineskip 27pt minus 1pt}
\def\secfont{\twelvepoint\bf\baselineskip 27pt minus 1pt}
\def\seccfont{\twelvepoint\bf\baselineskip 27pt minus 1pt}
\def\secccfont{\twelvepoint\bf\baselineskip 27pt minus 1pt}

\def\dotocnum#1{%
  \leavevmode 
     {\ifnonum \global\advance\nonumnum by1 \edef\thetocnum{}\fi
      \wlabel\thetocnum \dest[toc:\thetocnum]%
      \dotocnumafter}\ifnonum\else#1\fi
  \global\let\dotocnumafter=\relax
}

\def\printchap#1{\par \norempenalty-400 \bigskip
  {\chapfont \noindent 
   \ifnonum
      {\dotocnum{}{\raggedcenter{#1}\nbpar}}
    \else
      \dotocnum{\thetocnum\quad}#1\nbpar\fi}%
    \mark{}%
  \nobreak \remskip\bigskipamount \firstnoindent
}
\def\printsec#1{\par \norempenalty-400 \bigskip
  {\secfont \noindent 
   \ifnonum
      {\dotocnum{}{\raggedcenter{#1}\nbpar}}
    \else
      \dotocnum{\thetocnum\quad}#1\nbpar\fi}
    \insertmark{#1}%
  \nobreak \remskip\medskipamount \firstnoindent
}

\newcount\secccnum

\def\chaphook#1\relax{\secnum=0 \seccnum=0 \secccnum=0 \relax}
\def\sechook#1\relax{\seccnum=0 \relax}

\def\printseccc#1{\par \norempenalty-200 \medskip
  {\secccfont \noindent \dotocnum{\thetocnum\quad}#1\nbpar}%
  \nobreak \remskip\medskipamount \firstnoindent
}
\def\seccc#1\par{\ifnonum\else \global\advance\secccnum by1 \fi
  \edef\thesecccnum{\othe\chapnum.\the\secnum.\the\seccnum.\the\secccnum}\let\thetocnum=\thesecccnum
  \def\dotocnumafter{\wcontents\Xseccc{#1}}%
  \printseccc{#1\unskip}\resetnonumnotoc
}

%% Adicionado na versao beta(x) May 2014 do CTUstyle 
\def\begitemstest#1{\ifnum\catcode`\*=13
   \message{WARNING: \string#1 used in \string\begitems...\noexpand\enditems
            environment (line: \the\inputlineno)}\fi}%

%%% Appendicies

\def\appendixpage{%
    \cleardoublepage
	  \bgroup
	       \null\vfil
               {\nonum\chap \mtext{appendixpl}\par}
	       \vfil
	  \par\egroup
    \cleardoublepage
}%

\newcount\appnum
\def\appletter{\ifcase\appnum ?\or A\or B\or C\or D\or E\or F\or G\or H\or
   I\or J\or K\or L\or M\or N\or O\or P\or Q\or R\or S\or T\or U\or V\or
   W\or X\or Y\or Z\else ?\fi}

\def\app#1\par{\global\advance\appnum by1
  \ifx\chap\nochap \else \nextoddpage \global\let\chap=\nochap \fi
  \secnum=0 \seccnum=0 \relax
  \edef\theappnum{\appletter}\let\thechapnum=\theappnum \let\thetocnum=\theappnum
  \gdef\sechook ##1\def{\global\seccnum=0 
       \edef\thesecnum{\theappnum.\the\secnum}\let\thetocnum=\thesecnum
       \def}%
  \gdef\secchook ##1\def{%
       \edef\theseccnum{\theappnum.\the\secnum.\the\seccnum}\let\thetocnum=\theseccnum
       \def}%
  \def\dotocnumafter{\wcontents\Xchap{#1}}% 
  \printapp{#1\unskip}%
  \nobreak
}
\def\nochap#1\par{\message{CTUstyle WARNING: \noexpand\chap inside
                           Appendices is ignored.}}

\def\printapp#1{%
     \bgroup%\begitemstest\app%
     \tenpoint\tensingle \noindent%
     \settowidth{\mylen}{\mtext{appendix}\ \theappnum\ -\enspace}%
     \leftskip=\mylen%
     \printcaption{\mtext{appendix}}{\theappnum}#1\par\nobreak\egroup%
     \wlabel\theappnum% para poder usar \label
  \bigskip%
  \firstnoindent%
}

\def\bibchap{{\flushgluel=0pt \nonum \chap \mtext{bibliography}\par}}

%%% Captions

\def\thetnum{\the\tnum}
\def\thefnum{\the\fnum}

\def\caption/#1 #2{\isdefined{#1num}%
   \iftrue \global\advance \csname #1num\endcsname by1%
   \else   \opwarning{Unknown caption /#1}%
   \fi%
   \bgroup
      \leftskip=\iindent plus1fil
    %  \rightskip=\iindent plus-1fil  %recuo a direita. Acho que as normas da ESALQ nao pode ter esse recuo
      \parfillskip=0pt plus2fil
      \def\par{\endgraf\egroup}%
      \captionhook{#1}{#2}\noindent 
      \wlabel{\csname the#1num\endcsname}%
      \printcaption{\mtext{#1}}{\csname the#1num\endcsname}#2\par%
}

\def\captionhook#1#2{%
   \tenpoint\tensingle%
   \settowidth{\mylen}{\mtext{#1}\ {\csname the#1num\endcsname}\ -\enspace}
   \leftskip=\mylen %
%   \ifx\clabeltext\undefined \else
%      \toks0=\expandafter{\clabeltext}%
      \toks0=\expandafter{#2}%
      \ifx#1t\edef\tmp{\noexpand\wref\noexpand\Xtab
                       {{\lastlabel}{\thetnum}{\the\toks0}}}\tmp
      \else  \edef\tmp{\noexpand\wref\noexpand\Xfig
                       {{\lastlabel}{\thefnum}{\the\toks0}}}\tmp
   \fi%\fi
  % \global\let\clabeltext=\undefined
   \vskip-\parskip
}

%\def\printcaption#1#2{{\bf#1 #2.}\enspace}
\def\printcaption#1#2{\llap{#1 #2 -\enspace}}

%\def\clabel[#1]#2{\gdef\clabeltext{#2}\label[#1]}

\def\tofline#1#2#3#4{{\leftskip=0pt \rightskip=\iindent plus1em
   \ctpush{\hbox to 5em{\mtext{#4}\ \ref[#1]\ -}}
   {#3\unskip}\nobreak\tocdotfill\pgref[#1]\nobreak\hskip-\iindent\null\par}}
\let\totline=\tofline

\def\maketof{\par \ifx\toflist\empty
      \opwarning{\noexpand\maketof -- data unavailable, TeX me again}\openref
   \else \toflist \fi}

\def\maketot{\par \ifx\totlist\empty
      \opwarning{\noexpand\maketot -- data unavailable, TeX me again}\openref
   \else \totlist \fi}

%%% Numbered paragraphs

\newcount\numA \newcount\numB \newcount\numC \newcount\numD \newcount\numE

%\def\chaphook#1\relax{\secnum=0 \seccnum=0 \secccnum=0 \relax}

\def\chaphook#1\relax{\numA=0 \numB=0 \numC=0 \numD=0 \numE=0 
   \secnum=0 \seccnum=0 \secccnum=0 \relax}

\def\numberedpar#1#2{\par \global\advance\csname num#1\endcsname by1
   \noindent\wlabel{\thechapnum.\the\csname num#1\endcsname}%
   {\bf#2 \thechapnum.\the\csname num#1\endcsname.}\space} 

%%% Blue verbatim

%\ttindent=\parindent
\ttindent=4.1mm

%{\tenss \thefontscale[700] \global\let\sevenss=\thefont}

\def\tthook{\parskip=0pt %\typosize[10.5/13.6]
   \tenpoint\setbaselineskip[13.6]
}

\def\begtt{\ttskip\vskip.5\parskip\bgroup \aftergroup\parskipcorr\wipeepar
   \setverb \adef{ }{ }%
   \ifx\savedttchar\undefined \else \catcode\savedttchar=12 \fi
   \parindent=\ttindent
   \tthook\relax
   %\everypar={\rlap{\locc\liBlue 
   \everypar={\rlap{\locc\liGrey  % muda de azul pra cinza 
    \hskip-\ttindent \vrule width\hsize \strut}%
    \ifnum\ttline<0 \else \global\advance\ttline by1
                \llap{\locc\Blue\sevenss\the\ttline\kern.5em\indent}\fi \kern2pt\Black}
   \def\par##1{\endgraf\ifx##1\egroup\else\penalty\ttpenalty\vskip-1pt\leavevmode\fi ##1}
   \obeylines \startverb
}
\def\parskipcorr{\vskip-.5\parskip}

\def\viprintline{\vskip-1pt\indent
   \rlap{\locc\liBlue \hskip-\ttindent \vrule width\hsize \strut}%
   \ifnum \ttline<-1 \else 
      \llap{\locc\Blue\sevenrm\ifnum\ttline<0 \the\viline \else
               \global\advance\ttline by1 \the\ttline \fi \kern.5em\indent}\kern2pt
   \fi
   \Black \tmp\par % print the line from \tmp
}


%%% Blue centered tables

\def\ctable#1#2{
   \smallskip\centerline{\setbox0=\table{#1}{#2}%
   %\rlap{\locc\liBlue \tmpdim=\ht0 \advance\tmpdim by3pt
   \rlap{\locc\White \tmpdim=\ht0 \advance\tmpdim by3pt % alterado pra ficar sem cor azul 'liBlue'
   \vrule width\wd0 height\tmpdim depth5pt\Black}\box0}\nobreak
}
\def\tabiteml{\indent}\def\tabitemr{\indent}

\def\cinspic#1 {\centerline{\inspic #1 }\nobreak\medskip}
\let\oriendinsert=\endinsert
\def\endinsert{\par\oriendinsert}
%\def\tabiteml{\kern\iindent} \def\tabitemr{\kern\iindent}
\def\tabiteml{\kern 4.1mm} \def\tabitemr{\kern 4.1mm}


%%%%%%%%%%%%%% \fnote

% modificando a funcao vfootnote e textindent (e renomeando) do plain.tex pra eliminar o espaco entre o indice e o texto
\def\etextindent#1{\indent\llap{#1}\ignorespaces}%

\catcode`@=11
\def\evfootnote#1{\insert\footins\bgroup%
  \interlinepenalty\interfootnotelinepenalty%
  \splittopskip\ht\strutbox% top baseline for broken footnotes
  \splitmaxdepth\dp\strutbox \floatingpenalty\@MM%
  \leftskip\z@skip \rightskip\z@skip \spaceskip\z@skip \xspaceskip\z@skip%
  \etextindent{#1}\footstrut\futurelet\next\fo@t}%
\catcode`@=12

\def\fnote#1{\global\advance \fnotenum by1
   \iflocfnum \leavevmode\openref\wref\Xfnote{}%
      \isdefined{fn:\the\fnotenum}\iftrue
      \else\opwarning{unknown \noexpand\fnote mark. TeX me again}\fi\fi
   \fnmarkx{\fnotehook\tenpoint\tensingle\evfootnote\fnmarkx{#1}}%
}
\def\fnotetext#1{\global\advance \fnotenum by1 \openref\wref\Xfnote{}%
   {\tenpoint\tensingle\evfootnote\fnmarkx{#1}}%
}
% retira um parentese colocado na nota de rodape
\def\thefnote{$^{\locfnum}$}%


%%% Golossary (OPmac trick 0051, 0053, 0054

\newdimen\maxglosindent
\def\gloslist{}
\def\glos #1#2{\def\tmpb{#1}%
   \expandafter\isinlist\expandafter\gloslist\csname;\mmeaning\tmpb\endcsname
   \iftrue \opwarning{Duplicated glossary item `#1'}%
   \else
      \global\sdef{;\mmeaning\tmpb}{{#1}{#2}}%
      \global\expandafter\addto\expandafter\gloslist\csname;\mmeaning\tmpb\endcsname
   \fi
}
\def\makeglos{%
   \ifx\gloslist\empty \opwarning{Gossary data unavailable}%
   \else
      \bgroup
        \let\iilist=\gloslist
        \preparespecialsorting \dosorting
        \ifdim\maxglosindent=0pt 
           \ifx\glosindent\undefined \maxglosindent=2\parindent
              \edef\glosindent{\the\maxglosindent}\fi
        \else \edef\glosindent{\the\maxglosindent}\fi
        \maxglosindent=0pt
        \def\act##1{\ifx##1\relax \else\expandafter\printglos##1\expandafter\act\fi}
        \expandafter\act\iilist\relax
        \immediate\write\reffile{\string\def\string\glosindent{\the\maxglosindent}}
      \egroup
   \fi
}
\def\preparespecialsorting{%
   \setprimarysorting
   \def\act##1{\ifx##1\relax\else \preparesorting##1%
      \expandafter\edef\csname\string##1\endcsname{\tmpb}%
      \expandafter\act\fi}%
   \expandafter\act\iilist\relax
   \def\firstdata##1{\csname\string##1\endcsname&}%
}
\def\glosref #1#2{\if^#2^\else \glos{#1}{#2}\fi 
   \makegloslink{#1}\link[glos:\tmp]{}{#1}}
\def\glref #1{\glosref{#1}{}}
\let\glosborder=\tocborder

\def\printglos#1#2{%
   \setbox0=\hbox{#1}\ifdim\wd0>\maxglosindent \maxglosindent=\wd0 \fi
   \noindent \hangindent=\glosindent \advance\hangindent by2em
      \hbox to\glosindent{\makegloslink{#1}\dest[glos:\tmp]#1\hss}%
      \hbox to2em{\hss\normalitem\hss}#2\par}

\def\makegloslink#1{\def\tmpb{#1}\edef\tmpb{\mmeaning\tmpb}%
   \def\tmp{}\expandafter\makegloslinkA\tmpb\relax}
\def\makegloslinkA#1{\ifx#1\relax\else
   \edef\tmp{\tmp\number`#1.}\expandafter\makegloslinkA\fi}

\def\mmeaning#1{\expandafter\mmeaningA\meaning#1\relax}
\def\mmeaningA#1->#2\relax{#2}


%%% Requests for corrections (OPmac trick 0056)

\newcount\rfcnum

\def\rfclist{}  
\def\rfcactive#1{\global\advance\rfcnum by1
   \dest[rfc:rfc-\the\rfcnum]\global\addto\rfclist{\rfcitem #1}}
\def\rfc#1{}
\def\rfcitem{\advance\rfcnum by1
   \medskip\noindent\llap{\link[rfc:rfc-\the\rfcnum]{\localcolor\Red}{[rfc-\the\rfcnum]} }}

%\addto\draft{\let\rfc=\rfcactive}  %% uncomment this line if you need this feature

\def\makerfc{\ifx\rfc\rfcactive
   \vfil\break {\secfont \noindent Requests for correction}\par
   \bgroup\rfcnum=0 \rfclist\egroup\fi}

\let\rfcborder=\tocborder

\outer\def\bye{\par\vfill\supereject\makerfc\end}


%%% Items

%\def\normalitem{{\locc\Grey\bf\thefontsize[35]\raise.5pt\hbox{.}\kern-1pt}\enspace}
\def\normalitem{\locc\Grey{\bf\thefontsize[35].\kern-4pt}\Black\enspace}
\sdef{item:x}{\raise.4ex\hbox{\locc\Grey\bf\thefontsize[17].\Black} }
\sdef{item:n}{\the\itemnum.\kern.25em }
\addto\begitems{\vskip.5\parskip \parskip=0pt\relax}
\addto\enditems{\vskip-.5\parskip}

\def\bull{\vrule height .9ex width .8ex depth -.1ex } % Marcador de item (quadrado solido)

%\def\normalitem{\locc\Grey\bull\Black\enspace}


%%% BibItems

\let\alf\relax
\input bib/abntcite-plain.tex %Carrega os padroes da ABNT de citacao
\citeoption{abnt-repeated-title-omit=yes, abnt-show-options=warn, abnt-emphasize=bf, abnt-etal-list=0, abnt-url-package=url}

\let\oriurl=\url

\def\urlfont{\rm \hyphenchar\the\font=-1 \let\|=\urlspecchar}

\def\bibtexhook{%
   \def\preurl{}%
   \def\url##1{<\oriurl{##1}>}
}

\def\makebib#1{% criei essa funcao adicional
   \cleardoublepage
   \bibchap\bigskip
   \twelvesingle
   \raggedright\nohyphens
   %\bibliographystyle{mystyle} % caso queira usar outro estilo .bst no lugar do abnt
   \def\bibskip{\bigskip}  % space between bibitems
   \bibliography{#1} % listar as bases .bib usadas, entre virgulas. 
}                           % A abntex2-options.bib eh adicionada automaticamente 


%%% Last thinks:

\def\abbrv[#1]{\par \noindent\llap{#1\quad}\ignorespaces}

\def\urlnote#1{\fnote{\url{#1}}}
\def\nextoddpage{\vfil\supereject
   \ifodd\pageno \else \shipout\null \advancepageno \fi}

\addto\runningfnotes{\addto\chaphook{\global\fnotenum=0}}

\addprotect\cite
\addprotect\citeonline

\def\dprime{"}
\activettchar"

\endinput

%%% Versions:

beta(a) Jul.2014  (06/07/2014)
  (according to CTUstyle beta(u) Apr.2004)

beta(b) - First version released
          \app/secchook corrected
          \glos, \glosref, \makeglos introduced
          \rfc introduced
          Bug removing: labels with "_" can be printed in \draw mode
          \addprotect\cite plus \clabel correction
  (according to CTUstyle beta(Z) Oct.2014)
