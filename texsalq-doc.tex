% The documentation of the usage of TeXSALQ -- the template for
% typessetting thesis by plain\TeX at ESALQ - USP
% ---------------------------------------------------------------------
% Luciano R Silveira  Jun.-Jul. 2014
% lrsilveira@gmail.com
%
% Based on ctustyle-doc.tex by Petr Olsak

\input macros/texsalq   % The template is included here.

\worktype [O/PT] % Type: M = master, D = Ph.D., O = other
                 % / the language: PT = Portuguese, EN = English

\title      {\TeX \kern -.1667em\lower .5ex\hbox{SALQ} -- Instru��es de uso: modelo em \TeX{} puro para digita��o de disserta��es e teses na ESALQ/USP}
\author     {Luciano Roberto da Silveira / Andre Herman Freire Bezerra}
\authorinfo {Bacharel em Ci�ncias dos Alimentos / Bacharel em Engenharia Agron�mica}
\city       {Piracicaba}
\date       {2014}
\supervisor {Prof. Dr. \bf\uppercase{Nome do Orientador}}  % One or more supervisors
\workname   {Documenta��o do modelo desenvolvido em \TeX{} puro para disserta��es e teses da ESALQ, baseado no modelo CTUstyle, de Petr Ol\v{s}�k} % Used only if \worktype [O/*] (Other)

            % Title / Subtitle in minor language:
\titleEN    {\TeX \kern -.1667em\lower .5ex\hbox{SALQ} -- the user manual (the plain\TeX{} template for theses at ESALQ/USP)}
\titarea    {Ci�ncias}
\concarea   {Engenharia de Sistemas Agr�colas}
\pagetwo    {}  % The text printed on the page 2 at the bottom.
\dedication {      %  Optional. Use main language here
     Ao passado, \par\vskip 0.3cm
    ao presente e \par\vskip 0.3cm
    ao futuro \par\vskip 1.3cm
    Com amor, \bfit DEDICO\par
}
\thanks     {      %  Optional. Use main language here
    A Jorge Alexandre Wiendl, por nos encorajar na utiliza��o do \TeX{} puro, dando o pontap� inicial e contribuindo com muitas dicas de programa��o para o modelo criado. O trabalho foi um grande desafio. % No in�cio achamos loucura, mas com a pr�tica, percebemos algumas vantagens sobre o \LaTeX{}, como a redu��o do n�mero de linhas de c�digo e o controle total de todos os detalhes do documento. O trabalho foi encarado como um grande desafio.

    A Ismael Meurer, o primeiro a utilizar o modelo antes da primeira vers�o estar conclu�da, incentivando a continua��o do trabalho e relatando as dificuldades iniciais. A ele e ao Fernando Thomazini, tamb�m utilizador do modelo, por terem encontrado bugs importantes, corrigidos posteriormente, possibilitando melhoria.

   A Petr Ol\v{s}�k, professor na Universidade T�cnica Checa em Praga, autor do conjunto de macros tomadas como base para esse trabalho; pelo pronto e valioso suporte com as macros utilizadas, corrigindo os bugs relatados.
}
\epigraph   {       %  Optional. Use main language here
   {\sl Toda a nossa ci\^encia, comparada com a realidade, \'e primitiva e infantil --- e, no entanto, \'e a coisa mais preciosa que temos.}\par\vskip 1.3cm
Albert Einstein
}
\abstractEN {%
   This document shows an usage of the plain\TeX{} officially
   (may be) recommended design style \TeX SALQ for master
   (Ing.), or doctoral (Ph.D.) theses at the ESALQ/USP. The template
   defines all thesis mandatory structural elements and
   typesets their content to fulfil the university formal rules.
}
\abstractPT {%
   Este documento mostra a utiliza��o do modelo \TeX SALQ em \TeX{} puro,
   (que poderia ser) oficialmente recomendado para disserta��es e teses na  ESALQ/USP. O modelo
    define todos os elementos estruturais obrigat�rios em disserta��es e teses e
    edita seu conte�do para cumprir as regras formais da universidade.
}

\keywordsEN {%
   document design template; master, Ph.D. thesis; \TeX{}.
}
\keywordsPT {%
  modelo de documento; disserta��o de mestrado; tese de doutorado; \TeX{}.
}


%%%%% <--   % The place for your own macros is here.
\def\emph#1{{\it #1}}
\def\ttb{\tt\char`\\} %  sequence control printing of Tables
\def\asp#1{``#1''} % facilitates the use of quotation marks

%\draft     % Uncomment this if the version of your document is working only.
%\linespacing=1.7  % uncomment this if you need more spaces between lines
                   % Warning: this works only when \draft is activated!
%\savetoner        % Turns off the lightBlue or lightGrey backround of
                   % verbatims, only for \draft version.
%\blackwhite       % Use this if you need really Black+White thesis.

\makefront  % Mandatory command. Makes title page, acknowledgment, contents etc.

% Files where the source of the document is prepared.
% Full name is: body.tex, appendicies.tex, the suffix can be omitted.

%\input epsf % Works onty if pdfoutput = 0
\input body

{\makebib{mybib}}

\input appendicies

\bye
